%%%%%%%%%%%%%%%%%%%%%%%%%%%%%%%%%%%%%%%%%%%%%%%%%%%%%%%%%%%%%%%%%%%%%%%%%%%%%%%
%% Name:        flexsizr.tex
%% Purpose:     wxFlexGridSizer
%% Author:      wxWidgets Team
%% Modified by:
%% Created:
%% RCS-ID:      $Id: flexsizr.tex 41905 2006-10-10 17:46:49Z JS $
%% Copyright:   (c) wxWidgets Team
%% License:     wxWindows license
%%%%%%%%%%%%%%%%%%%%%%%%%%%%%%%%%%%%%%%%%%%%%%%%%%%%%%%%%%%%%%%%%%%%%%%%%%%%%%%

\section{\class{wxFlexGridSizer}}\label{wxflexgridsizer}

A flex grid sizer is a sizer which lays out its children in a two-dimensional
table with all table fields in one row having the same
height and all fields in one column having the same width, but all
rows or all columns are not necessarily the same height or width as in
the \helpref{wxGridSizer}{wxgridsizer}.

Since wxWidgets 2.5.0, wxFlexGridSizer can also size items equally in one
direction but unequally ("flexibly") in the other. If the sizer is only
flexible in one direction (this can be changed using
\helpref{SetFlexibleDirection}{wxflexgridsizersetflexibledirection}),
it needs to be decided how the sizer should grow in the other ("non-flexible")
direction in order to fill the available space. The
\helpref{SetNonFlexibleGrowMode}{wxflexgridsizersetnonflexiblegrowmode} method
serves this purpose.

\wxheading{Derived from}

\helpref{wxGridSizer}{wxgridsizer}\\
\helpref{wxSizer}{wxsizer}\\
\helpref{wxObject}{wxobject}

\wxheading{Include files}

<wx/sizer.h>

\wxheading{See also}

\helpref{wxSizer}{wxsizer}, \helpref{Sizer overview}{sizeroverview}

\latexignore{\rtfignore{\wxheading{Members}}}

\membersection{wxFlexGridSizer::wxFlexGridSizer}\label{wxflexgridsizerwxflexgridsizer}

\func{}{wxFlexGridSizer}{\param{int }{rows}, \param{int }{cols}, \param{int }{vgap}, \param{int }{hgap}}

\func{}{wxFlexGridSizer}{\param{int }{cols}, \param{int }{vgap = 0}, \param{int }{hgap = 0}}

Constructor for a wxGridSizer. {\it rows} and {\it cols} determine the number of
columns and rows in the sizer - if either of the parameters is zero, it will be
calculated to form the total number of children in the sizer, thus making the
sizer grow dynamically. {\it vgap} and {\it hgap} define extra space between
all children.


\membersection{wxFlexGridSizer::AddGrowableCol}\label{wxflexgridsizeraddgrowablecol}

\func{void}{AddGrowableCol}{\param{size\_t }{idx}, \param{int }{proportion = $0$}}

Specifies that column {\it idx} (starting from zero) should be grown if
there is extra space available to the sizer.

The {\it proportion} parameter has the same meaning as the stretch factor for
the \helpref{sizers}{sizeroverview} except that if all proportions are $0$,
then all columns are resized equally (instead of not being resized at all).

\membersection{wxFlexGridSizer::AddGrowableRow}\label{wxflexgridsizeraddgrowablerow}

\func{void}{AddGrowableRow}{\param{size\_t }{idx}, \param{int }{proportion = $0$}}

Specifies that row idx (starting from zero) should be grown if there
is extra space available to the sizer.

See \helpref{AddGrowableCol}{wxflexgridsizeraddgrowablecol} for the description
of {\it proportion} parameter.

\membersection{wxFlexGridSizer::GetFlexibleDirection}\label{wxflexgridsizergetflexibledrection}

\constfunc{int}{GetFlexibleDirection}{\void}

Returns a wxOrientation value that specifies whether the sizer flexibly
resizes its columns, rows, or both (default).

\wxheading{Return value}

One of the following values:

\begin{twocollist}
\twocolitem{wxVERTICAL}{Rows are flexibly sized.}
\twocolitem{wxHORIZONTAL}{Columns are flexibly sized.}
\twocolitem{wxBOTH}{Both rows and columns are flexibly sized (this is the default value).}
\end{twocollist}

\wxheading{See also}

\helpref{SetFlexibleDirection}{wxflexgridsizersetflexibledirection}


\membersection{wxFlexGridSizer::GetNonFlexibleGrowMode}\label{wxflexgridsizergetnonflexiblegrowmode}

\constfunc{int}{GetNonFlexibleGrowMode}{\void}

Returns the value that specifies how the sizer grows in the "non-flexible"
direction if there is one.

\wxheading{Return value}

One of the following values:

\begin{twocollist}
\twocolitem{wxFLEX\_GROWMODE\_NONE}{Sizer doesn't grow in the non-flexible direction.}
\twocolitem{wxFLEX\_GROWMODE\_SPECIFIED}{Sizer honors growable columns/rows set with
\helpref{AddGrowableCol}{wxflexgridsizeraddgrowablecol} and
\helpref{AddGrowableRow}{wxflexgridsizeraddgrowablerow}.
In this case equal sizing applies to minimum sizes of columns or
rows (this is the default value).}
\twocolitem{wxFLEX\_GROWMODE\_ALL}{Sizer equally stretches all columns or rows
in the non-flexible direction, whether they are growable or not in the flexible
direction.}
\end{twocollist}

\wxheading{See also}

\helpref{SetFlexibleDirection}{wxflexgridsizersetflexibledirection},
\helpref{SetNonFlexibleGrowMode}{wxflexgridsizersetnonflexiblegrowmode}


\membersection{wxFlexGridSizer::RemoveGrowableCol}\label{wxflexgridsizerremovegrowablecol}

\func{void}{RemoveGrowableCol}{\param{size\_t }{idx}}

Specifies that column idx is no longer growable.


\membersection{wxFlexGridSizer::RemoveGrowableRow}\label{wxflexgridsizerremovegrowablerow}

\func{void}{RemoveGrowableRow}{\param{size\_t }{idx}}

Specifies that row idx is no longer growable.


\membersection{wxFlexGridSizer::SetFlexibleDirection}\label{wxflexgridsizersetflexibledirection}

\func{void}{SetFlexibleDirection}{\param{int }{direction}}

Specifies whether the sizer should flexibly resize its columns, rows, or
both. Argument {\tt direction} can be {\tt wxVERTICAL}, {\tt wxHORIZONTAL}
or {\tt wxBOTH} (which is the default value). Any other value is ignored. See
\helpref{GetFlexibleDirection()}{wxflexgridsizergetflexibledrection} for the
explanation of these values.

Note that this method does not trigger relayout.


\membersection{wxFlexGridSizer::SetNonFlexibleGrowMode}\label{wxflexgridsizersetnonflexiblegrowmode}

\func{void}{SetNonFlexibleGrowMode}{\param{wxFlexSizerGrowMode }{mode}}

Specifies how the sizer should grow in the non-flexible direction if
there is one (so
\helpref{SetFlexibleDirection()}{wxflexgridsizersetflexibledirection} must have
been called previously). Argument {\it mode} can be one of those documented in
\helpref{GetNonFlexibleGrowMode}{wxflexgridsizergetnonflexiblegrowmode}, please
see there for their explanation.

Note that this method does not trigger relayout.

